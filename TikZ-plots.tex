\documentclass[
extrafontsizes,
 twoside,
 11pt,
 openright,
 final,
 %draft
 ]{memoir}
\usepackage{livroaberto}
\usepackage{xcolor}
\usepackage{ctex}
\usepackage{tasks}
\usepackage{pgfplots}
\usetikzlibrary{
calc,
patterns,
intersections,
through,
angles,
backgrounds,
positioning
}
\usepgfplotslibrary{patchplots}
\usepackage{yhmath}
\usepackage[mathscr]{eucal}
\usetikzlibrary{math}
\usepackage{tkz-euclide}

\renewcommand{\lstlistlistingname}{Codeverzeichnis}
\renewcommand{\lstlistingname}{Quellcode}
% \lstset{caption=\lstname}
% Globale Einstellungen
\lstdefinestyle{latex}{
language=[AlLaTeX]TeX,
extendedchars=true,
frame=single,
keepspaces=true,
backgroundcolor=\color{shang1!8!white},
rulecolor=\color{white},
breaklines=true,
xleftmargin=\parindent,
basicstyle=\footnotesize\ttfamily,
basicstyle=\scriptsize\ttfamily\color{shang},
keywordstyle=\bfseries\color{blue},
commentstyle=\itshape\color{gray},
identifierstyle=\bfseries\color{shang},
stringstyle=\color{orange},
}
%=================================
\usepackage{accsupp}
\newcommand{\noncopynumber}[1]{ \BeginAccSupp{method=escape,ActualText={}}
#1
\EndAccSupp{}}

\lstset {numberstyle=\tiny\color{gray!90!black}\noncopynumber,
numbers=none,
%numbersep=2pt,
columns=flexible,
title=\bfseries
}


\definecolor{Green}{RGB}{0,165,157}
\definecolor{myback}{RGB}{53,64,89}
\definecolor{mycolor}{RGB}{51,51,51}
\definecolor{Orange}{RGB}{102,51,0}
\definecolor{Green}{RGB}{0,102,51}
\definecolor{Blue}{RGB}{15,107,108}
\definecolor{Yellow}{RGB}{153,151,51}
\definecolor{CUDRed}{RGB}{255,75,0}
\definecolor{CUDGreen}{RGB}{3,175,122}
\definecolor{CUDBlue}{RGB}{0,90,255}
\definecolor{CUDCyan}{RGB}{77,196,255}
\definecolor{CUDMagenta}{RGB}{153,0,153}
\definecolor{CUDYellow}{RGB}{255,241,0}
\definecolor{CUDBrown}{RGB}{128,64,0}
\definecolor{CUDOrange}{RGB}{246,170,0}
\definecolor{CUDPink}{RGB}{255,202,191}
\definecolor{CUDBrightGreen}{RGB}{119,217,168}
\definecolor{CUDLime}{RGB}{216,242,85}
\definecolor{CUDCream}{RGB}{255,255,128}
\definecolor{CUDBrightCyan}{RGB}{191,228,255}


\def\myparallel{
\setbox0\hbox{$=$}\makebox[\the\wd0][c]{\hss\(\slash\kern-0.2em\slash\)\hss}
}

\def\maruwaku#1{\begin{tikzpicture}[scale=0.7, baseline={([yshift=-22pt] current bounding box.north)}]
\filldraw[color=CUDBlue, line width=1pt, rounded corners=2pt] (-0.1,0)--(2.1,0)--(2.1,1.1)--(-0.1,1.1)--cycle;
\draw(1,0.5) node[whi te]{#1};
\end{tikzpicture}
}

\DeclareMathOperator{\ee}{\!\!\;\mathrm e}
\newcommand{\MR}{\mathbb R}
\newcommand{\MC}{\mathbb C}
\newcommand{\MF}{\mathbb F}
\newcommand{\MZ}{\mathbb Z}
\newcommand{\MN}{\mathbb N}
\newcommand{\MCF}{\mathscr F}
\newcommand{\ii}{\mathrm i}

\usepackage{shapepar}

\begin{document}

\ifnum \aluno=1
\renewcommand\chapterillustration{chapter-1}
\else
\renewcommand\chapterillustration{abertura-funcoes-professor}
\fi
\renewcommand\chapterwhat{由于各种原因,上期上传的TikZ绘图资源:\href{https://www.latexstudio.net/index/details/index/mid/2025.html}{高中数学TikZ 绘图合集}后有两部分内容没有完成,现在将空间几何部分的内容上传于此文档,内容编辑和搜集全程本人来做,若有问题请反馈至我的邮箱:yonguel487@qq.com.}
\renewcommand\chapterbecause{本部分内容主要是空间立体几何的图形绘制,高中数学教师下载学习颇有价值,这里强烈推荐。绘制的图形来自于2019--2021年全国高考真题卷的空间集合试题,考虑到严肃性和规范性,绘制过程中几乎没有采用彩色绘制、填充等。笔记来自于之前的模板:\href{https://www.latexstudio.net/index/details/index/mid/1083.html}{TikZ 数表数据科研绘图}。}
\chapter{{\boxtitlefont TikZ} 空间立体几何}
\label{chap-1}

\mbox{}\thispagestyle{empty}

\thispagestyle{empty}
\def\funcoeschap{}

\mainmatter

\explore{高中立体几何图形的TikZ实现}

\indent 由于各种原因,上期上传的TikZ绘图资源:\href{https://www.latexstudio.net/index/details/index/mid/2025.html}{高中数学TikZ 绘图合集}后有两部分内容没有完成,现在将空间几何部分的内容上传于此文档,内容编辑和搜集全程本人来做,若有问题请反馈至我的邮箱:yonguel487@qq.com.

\indent 本部分内容主要是空间立体几何的图形绘制,高中数学教师下载学习颇有价值,这里强烈推荐。绘制的图形来自于2019--2021年全国高考真题卷的空间集合试题,考虑到严肃性和规范性,绘制过程中几乎没有采用彩色绘制、填充等。笔记来自于之前的模板:\href{https://www.latexstudio.net/index/details/index/mid/1083.html}{TikZ 数表数据科研绘图}。

\lstinputlisting[style=latex, firstline=10,lastline=23]{2019-Ia-18.tex}

\begin{minipage}{0.4\linewidth}
  %\begin{research}
  \includegraphics{2019-beijing-16}
  %\end{research}
\end{minipage}\hfill
\begin{minipage}{0.56\linewidth}
  %\begin{knowledge}
  \lstinputlisting[style=latex, firstline=10,lastline=23]{2019-Ia-18.tex}
  %\end{knowledge}
\end{minipage}\hfill

\includegraphics{2019-beijing-16}

\includegraphics{2019-beijingw-18}

\includegraphics{2019-Ia-12}

\includegraphics{2019-Ia-18}

\includegraphics{2019-Ia-18da}

\includegraphics{2019-Ib-16}

\includegraphics{2019-Ib-18}

\includegraphics{2019-Ib-18da}




\input{chapters/chap1.tex}




\end{document}
